
\mysection{Langages utilisés}

	\mysubsection{HTML}
		HTML est un langage qui permet de structurer une page web à l’aide de marqueurs, balises ou tags. Une balise HTML est représentée sous la forme ‘<\ldots>’  ‘</\ldots>’ ouvrante et fermante. Elle peut être de plusieurs types. Par exemple, si l’on souhaite créer un paragraphe, on mettra ‘<p> texte </p>’. J’ai utilisé toutes les balises HTML à l’intérieur des balises PHP.

	\mysubsection{CSS}
		Le CSS permet la mise en forme d’une page web. Le choix des polices, des couleurs, de l’habillage des tableaux, et le placement de ces derniers sont déterminés grâce à ce langage. Il est inclut dans les balises HTML sous la forme d’identifiants ou de classe. Ils sont tous les deux interprétés coté client.

	\mysubsection{MySQL}
		Il s'agit d'un langage qui premet de faire des requêtes sur une base de données

	\mysubsection{PHP}
		Il s'agit d'un langage interprété côté serveur. Il permet de rendre une page dynamique grâce à son intéraction entre la base de données et le HTML.\\
		\mybox{
			\lstinputlisting[caption={Un hello world},
			  				label=hello,
			  				tabsize=2,
			  				language=PHP]{./tex/code/hello.php}
			}


	\mysubsection{JavaScript}
		C'est en langage interprété côté client. C'est le navigateur Web qui se charge de l'exécution du programme. Il permet d'intéragir avec le HTML et le CSS et rendre une page Web dynamique.\\

		\mybox{
			\lstinputlisting[caption={Un hello world JS} ,
			  				label=hello,
			  				tabsize=2,
			  				language=JavaScript]{./tex/code/hello.js}
			}


	\mysubsection{MOOTOOLS}
		Tout comme JQuery, il s'agit un framework JavaScript libre qui offre la possibilité d'écrire moins, et plus facilement de code dans le script en étant au moins aussi puissant. Il offre des possibilités qui seraient complexes à mettre en oeuvre en javaScript.
		\mybox{
			\lstinputlisting[caption={Un hello world MooTools},
			  				label=hello,
			  				tabsize=2,
			  				language=JavaScript]{./tex/code/hello2.js}
			}
	

	\mysubsection{AJAX}
		Contrairement au PHP, l'AJAX permet de faire des modifications sur une partie de la page Web affichée et non la totalité de cette derniére. Son développement se fait par MooTools.


% \mybox{
% %\begin{boxedminipage}[poslb]{15.5cm}
% \lstinputlisting[caption={Un hello world},
%   				label=hello,
%   				tabsize=2,
%   				%backgroundcolor=yellow,
%   				language=PHP]{./tex/code/hello.php}
% %\end{boxedminipage}
% }
% comme le montre \ref{hello}
%\begin{lstlisting}[language=PHP]
%\lstinputlisting{hello.php}
%\end{lstlisting}

